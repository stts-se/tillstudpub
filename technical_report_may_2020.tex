% -*- latex -*-

\documentclass[11pt, a4paper, twoside]{article}
\usepackage{a4wide}
%\usepackage{a4}
%--------------------- Swedish ------------------------
\usepackage[T1]{fontenc}
\usepackage[latin1]{inputenc} %% Latin-1 / ISO-8859-1
% \usepackage[utf8]{inputenc} %% UTF-8
%\usepackage[swedish]{babel}
%-------------------------------------------------------

%% For att kunna ange totalt antal sidor, anvand paketet lastpage
%% nedan, samt efter \begin{document} lagg till:
%% \cfoot{\thepage (\pageref{LastPage})}
\usepackage{lastpage} 

%% \numberwithin{figure}{section}
%% \numberwithin{table}{section}
%% \setlength{\parindent}{0 pt}
%% \setlength{\parskip}{6 pt}

%% if you need to change page margins:
%% \setlength{\topmargin}{-1cm}
%% \setlength{\textheight}{24cm}

% Unix:
% Ctrl-c-f to make a dvi file of this
% Ctrl-c-v to open it in XDVI

\usepackage{fancybox}

\usepackage{fancyhdr}

\makeatletter
\renewcommand{\subsubsection}{\@startsection
  {subsubsection}%
  {3}%
  {0mm}%
  {-\baselineskip}%
  {0.5\baselineskip}%
  {\bfseries\sffamily}}%
\renewcommand{\subsection}{\@startsection
  {subsection}%
  {2}%
  {0mm}%
  {-\baselineskip}%
  {0.5\baselineskip}%
  {\bfseries\sffamily\large}}%
\renewcommand{\section}{\@startsection
  {section}%
  {1}%
  {0mm}%
  {-\baselineskip}%
  {0.5\baselineskip}%
  {\bfseries\sffamily\Large}}%
\makeatother


\pagestyle{fancy}
\fancyhf{}

\newcommand{\stts}{STTS S\"oder{\-}malms tal{\-}teknologi{\-}service}

\lhead{\small \sc \stts }

\rhead{\small stts.se}


%\usepackage{amssymb}
\title{ Microphone capture in the web browser }


\author{ \stts \\
  Folkungagatan 122 2tr, 116 30 Stockholm\\
  http://stts.se }

%-------------------------------------------------------
%---------------------- DOCUMENT -----------------------
%-------------------------------------------------------

\begin{document}

\maketitle
%  \cfoot{\thepage}
\cfoot{\thepage (\pageref{LastPage})}  


\section{Introduction}

In this document, we describe the background for some demo
applications using the web audio API for microphone recording in the
browser.  Different methods for recording in the browser and sending
the resultng audio to a server are presented.

There are different ways to capture the audio in the browser, and
different ways to transporting the audio to a server.

\bigskip

TODO: List the different methods explained below 


\section{Recording a complete file before sending to the server}

One way to do microphone recording in a web browser is using the
JavaScript Web Audio API. You can obtain an audio stream from the
microphone using the {\tt getUserMedia()} function. Using a
{\tt MediaRecorder}, the audio can be saved as a Blob. When the recording
is done, the Blob can be added to an HTML5 audio element, that can be
played in the web browser.

The audio file can be sent to a HTTP server (typically as a base64 encoded
string), and saved on the server.

This method works when the user wants to record something, and only
when done send it to the server. This could, for example, be a
recording tool, where you read aloud and record a manuscript sentence
presented in the browser, such as the current {\tt promprec} demo.

However, this method is not useful for streaming audio to the server.

\subsection*{Pros}
\begin{itemize}
\item The audio format is known
\item A complete audio file is created before sending it over the network
\end{itemize}

\subsection*{Cons}
\begin{itemize}
\item The audio format cannot be changed (it is controlled by the browser)
\item Not for streaming
\end{itemize}

\section{Streaming over a websocket}

A websocket can be thought of as a bi-directional HTTP
connection. Usually, when a client calls an HTTP server, there is a
request from the client, and a response from the server, and that's
it. Subsequent calls must establish new connections to the server.

A websocket, on the other hand, is an HTTP connection that may stay
open for any period of time, and on which both the client and the
server may send data. Since a websocket is an upgraded HTTP
connection, is has much of the same characteristics, such as
guarantees against package loss, etc.

A difference between a websocket and a normal HTTP connection, is that
there is not a well defined protocol for interaction between client
and server (such as GET and POST, etc). After a websocket connection
has been established, the client and the server may send text or
binary data to each other in any form.


\section{Streaming using WebRTC}

WebRTC (Web Real-Time Communication) \cite{webrtc} is a peer-to-peer method for
streaming audio and video. In other words, you can use it to stream
audio and video directly between web-browsers (as long as there is
some way of the browsers to find each other, for example using a STUN
server).

WebRTC is build on UDP (rather than HTTP). The WebRCT uses the Opus
codec for audio (and V8 for video).

Since human cognition is more forgiving to missing samples than to
latency, as little lag as possible is more important than being sure
that the original sound wave is complete and intact --- packet loss is
tolerated. The original sound wave might not be possible to reproduce
on the other end. Since speech is full of redundant information and
human cognition is built for interpreting a noisy signal, this may not
be a big issue for understanding speech sent over a bad network using
WebRTC.

Under the hood, WebRTC takes care of things like echo-canceling and
noise reduction. 


WebRTC also includes a DataChannel API, that can be used for
transporting data losslessly between peers. However, a websocket may
do the job equally good, if you do not need peer-to-peer capabilities.

\subsection*{Pros}
\begin{itemize}
\item Supported by most browsers \cite{webrtcsupport}
\item \ldots
\end{itemize}

\subsection*{Cons}
\begin{itemize}
\item Risk of packet loss
\item Little risk of latency
\item \ldots
\end{itemize}

\section{Streaming using ScriptProcessorNode}

The ScriptProcessorNode\cite{scriptprocessornode} was introduced to meet developers' need to process audio streams in the Web Audio API. Unlike other parts of the Web Audio API, the processing is run in the main thread, which can cause delays. It has since been deprecated. More information on the motivation behind the move to AudioWorklet can be found in \cite{icmc}.

\subsection*{Pros}
\begin{itemize}
\item Supported by most browsers
\item \ldots
\end{itemize}

\subsection*{Cons}
\begin{itemize}
\item Risk of packet loss and latency
\item Deprecated
\item \ldots
\end{itemize}


\section{Streaming using AudioWorklet}

The AudioWorklet was developed to handle some critical design flaws in the ScriptProcessorNode. The first implementation of the AudioWorklet was released 2018 for Chrome \cite{icmc}.

Unfortunately, the AudioWorklet is yet not fully supported by Firefox (neither stable version 75 nor beta version 76 work as of April 29th, 2020). According to the documentation, it should work with Firefox (stable) version 76. It works fine with Google Chrome (tested using version 81) and Opera (tested using version 68).


\subsection*{Pros}
\begin{itemize}
\item Processing run in a separate thread
\item Less latency than ScriptProcessorNode
\item \ldots
\end{itemize}

\subsection*{Cons}
\begin{itemize}
\item Not supported by all browsers (see \cite{audioworklet})
\item \ldots
\end{itemize}




\section{Browser settings for audio and streaming}

There seems to be some settings in the browser(s), that are difficult to control. Examples:

\begin{description}
\item[Sample rate] Can be retrived but not changed
\item[Channel count] We can set this value, but are not sure how it's used. The wav library currently used on the server side only supports mono.
\item[Audio encoding]\ \\[-14pt]
  \begin{description}
  \item[\em{AudioWorklet/ScriptProcessorNode}] For streaming, a browser's recording is by definition 32bit PCM. We have chosen to convert to 16bit PCM before sending to server.
  \item[\em{WebRTC}] OPUS
  \item[\em{ Non-streaming}] The browser controls the encoding, but the value can be retreived.
    
  \end{description}

\end{description}



\begin{thebibliography}{9}

\bibitem{highperfnetworking}
  \textit{High Performance Browser Networking},
  Ilya Grigorik,
  O'Reilly,
  2013
  
\bibitem{scriptprocessornode}
  https://developer.mozilla.org/en-US/docs/Web/API/ScriptProcessorNode

\bibitem{audioworklet}
  https://developer.mozilla.org/en-US/docs/Web/API/AudioWorklet

\bibitem{icmc}
  \textit{AudioWorklet: The future of web audio},
  Hongchan Choi,
  Google Chrome,
  ICMC,
  2018,
  https://hoch.io/media/icmc-2018-choi-audioworklet.pdf

\bibitem{webaudioapi}
  Boris Smus,
  \textit{Web Audio API},
  O'Reilly,
  2013

\bibitem{webrtc}
  https://webrtc.org/
  
\bibitem{webrtcsupport}
  https://en.wikipedia.org/wiki/WebRTC\#Support

\bibitem{}
  \textit{Investigation into low latency live video streaming performance of WebRTC},\newline
  Jakob Tidestr�m,
  In degree project computer science and engineering, second cycle, \newline
  Stockholm,
  Sweden,
  2019,\newline
  http://www.diva-portal.se/smash/get/diva2:1304486/FULLTEXT01.pdf
  
\end{thebibliography}

\end{document}


