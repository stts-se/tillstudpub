% -*- latex -*-

\documentclass[11pt, a4paper, twoside]{article}
\usepackage{a4wide}
%\usepackage{a4}
%--------------------- Swedish ------------------------
\usepackage[T1]{fontenc}
\usepackage[latin1]{inputenc} %% Latin-1 / ISO-8859-1
% \usepackage[utf8]{inputenc} %% UTF-8
%\usepackage[swedish]{babel}
%-------------------------------------------------------

%% For att kunna ange totalt antal sidor, anvand paketet lastpage
%% nedan, samt efter \begin{document} lagg till:
%% \cfoot{\thepage (\pageref{LastPage})}
 \usepackage{lastpage} 

%% \numberwithin{figure}{section}
%% \numberwithin{table}{section}
%% \setlength{\parindent}{0 pt}
%% \setlength{\parskip}{6 pt}

%% if you need to change page margins:
%% \setlength{\topmargin}{-1cm}
%% \setlength{\textheight}{24cm}

% Unix:
% Ctrl-c-f to make a dvi file of this
% Ctrl-c-v to open it in XDVI

\usepackage{fancybox}

\usepackage{fancyhdr}

\makeatletter
 \renewcommand{\subsubsection}{\@startsection
   {subsubsection}%
   {3}%
   {0mm}%
   {-\baselineskip}%
   {0.5\baselineskip}%
   {\bfseries\sffamily}}%
 \renewcommand{\subsection}{\@startsection
   {subsection}%
   {2}%
   {0mm}%
   {-\baselineskip}%
   {0.5\baselineskip}%
   {\bfseries\sffamily\large}}%
 \renewcommand{\section}{\@startsection
   {section}%
   {1}%
   {0mm}%
   {-\baselineskip}%
   {0.5\baselineskip}%
   {\bfseries\sffamily\Large}}%
\makeatother


\pagestyle{fancy}
\fancyhf{}

\newcommand{\stts}{STTS S\"oder{\-}malms tal{\-}teknologi{\-}service}

\lhead{\small \sc \stts }

\rhead{\small stts.se}


%\usepackage{amssymb}
\title{ Microphone capture in the web browser }


\author{ \stts \\
  Folkungagatan 122 2tr, 116 30 Stockholm\\
  http://stts.se }

%-------------------------------------------------------
%---------------------- DOCUMENT -----------------------
%-------------------------------------------------------

\begin{document}
  
  \maketitle
%  \cfoot{\thepage}
 \cfoot{\thepage (\pageref{LastPage})}  
  

In this document, we describe the background for the
`audio\_streaming` demo application, available methods for audio
streaming and technical challenges with different methods.


\section{Recording a complete file before sending to the server}

One way to do microphone recording in a web browser is using the
JavaScript Web Audio API. You can obtain an audio stream from the
microphone using the `getUserMedia()` function. Using a
`MediaRecorder`, the audio can be saved as a Blob. When the recording
is done, the Blob can be added to an HTML5 audio element, that can be
played in the web browser.

The audio file can be sent to a HTTP server as a base64 encoded
string, and decoded and saved on the server.

This method works when the user wants to record something, and only
when done send it to the server. This could, for example, be a
recording tool, where you read aloud and record a manuscript sentence
presented in the browser.

However, this method is not useful for streaming audio to the server.

\subsection*{Pros}
\begin{itemize}
\item The audio format is known
\item ...
  \end{itemize}

\subsection*{Cons}
\begin{itemize}
\item The audio format cannot be changed (it is controlled by the browser)
\item Not for streaming
\item ...
\end{itemize}

\section{Streaming over a websocket}

A websocket can be thought of as a bi-directional HTTP
connection. Usually, when a client calls an HTTP server, there is a
request from the client, and a response from the server, and that's
it. Subsequent calls must establish new connections to the server.

A websocket, on the other hand, is an HTTP connection that may stay
open for any period of time, and on which both the client and the
server may send data. Since a websocket is an upgraded HTTP
connection, is has much of the same characteristics, such as
guarantees against package loss, etc.

A difference between a websocket and a normal HTTP connection, is that
there is not a well defined protocol for interaction between client
and server (such as GET and POST, etc). After a websocket connection
has been established, the client and the server may send text or
binary data to each other in any form.


\section{Streaming using WebRTC}

WebRTC (Web Real-Time Communication) is a peer-to-peer method for
streaming audio and video. In other words, you can use it to stream
audio and video directly between web-browsers (as long as there is
some way of the browsers to find each other, for example using a STUN
server).

WebRTC is build on UDP (rather than HTTP). The WebRCT uses the Opus
codec for audio (and V8 for video).

Since human cognition is more forgiving to missing samples than to
latency, as little lag as possible is more important than being sure
that the original sound wave is complete and intact --- packet loss is
tolerated. The original sound wave might not be possible to reproduce
on the other end. Since speech is full of redundant information and
human cognition is built for interpreting a noisy signal, this may not
be a big issue for understanding speech sent over a bad network using
WebRTC.

Under the hood, WebRTC takes care of things like echo-canceling and
noise reduction. 


WebRTC also includes a DataChannel API, that can be used for
transporting data losslessly between peers. However, a websocket may
do the job equally good, if you do not need peer-to-peer capabilities.

\subsection*{Pros}
\begin{itemize}
\item Supported by most browsers (?)
\item ...
  \end{itemize}

\subsection*{Cons}
\begin{itemize}
\item Not "lossless" [TODO wording]
\item ...
\end{itemize}

\section{Streaming using ScriptProcessorNode}

Before the AudioWorket was introduced into the Web Audio API, the ScriptProcessorNode\cite{scriptprocessornode} could be used for streaming. This part of the Web Audio API has since been deprecated due to some critical design flaws. More information on the motivation behind the move to AudioWorklet can be found in \cite{icmc}.

Unfortunately, the AudioWorklet is not fully supported by Firefox yet (neither stable version 75 nor beta version 76 work as of April 29th, 2020). It works fine with Google Chrome (version 81) and Opera (version 68).

TODO more background

\subsection*{Pros}
\begin{itemize}
\item Supported by most browsers
\item ...
  \end{itemize}

\subsection*{Cons}
\begin{itemize}
\item Not "lossless" [TODO wording]
\item ...
\end{itemize}


\section{Streaming using AudioWorklet}

For streaming audio with JavaScript only (no external libraries) the Web Audio API and AudioWorlet\cite{audioworklet} is the ??? ??? method recommended by ??? ???.

TODO more background


\subsection*{Pros}
\begin{itemize}
\item Designed to be "lossless" [TODO wording]
\item ...
  \end{itemize}

\subsection*{Cons}
\begin{itemize}
\item Not supported by all browsers (see \cite{audioworklet})
\item ...
\end{itemize}




\section{Browser settings for audio and streaming}

There seems to be some settings in the browser(s), that are difficult to control or even access for reading. Examples:

\begin{itemize}
\item sample rate: we can read it, but haven't figure out how to change it
\item audio encoding: we currently cannot read or set this value in the browser
\item channel count: we can set this value, but we are not sure how it's used (may differ between ScriptProcessorNode and AudioWorklet)
\end{itemize}

We would like to be able to better control these settings, or at the very least find out what they are, so that we can confidently send this information to the server along with the other metadata. This information is needed to save the raw audio data as a specific audio file (wav or similar).


\begin{thebibliography}{9}

\bibitem{highperfnetworking}
  High Performance Browser Networking,
  Ilya Grigorik,
  O'Reilly,
  2013
  
\bibitem{scriptprocessornode}
  https://developer.mozilla.org/en-US/docs/Web/API/ScriptProcessorNode

\bibitem{audioworklet}
https://developer.mozilla.org/en-US/docs/Web/API/AudioWorklet

\bibitem{icmc}
  Hongchan Choi,
    \textit{AudioWorklet: The future of web audio},
    Google Chrome,
    ICMC,
    2018,
    https://hoch.io/media/icmc-2018-choi-audioworklet.pdf

\end{thebibliography}

\end{document}


